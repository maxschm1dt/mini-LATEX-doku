\documentclass{article}
\usepackage[ngerman]{babel}
\usepackage[left=2.5cm, right=2.5cm, top=2.5cm, bottom=2cm]{geometry}
\usepackage[onehalfspacing]{setspace}

\usepackage{listings}
\usepackage{xcolor}

\definecolor{codegreen}{rgb}{0,0.6,0}
\definecolor{codegray}{rgb}{0.5,0.5,0.5}
\definecolor{codepurple}{rgb}{0.58,0,0.82}
\definecolor{backcolour}{rgb}{0.95,0.95,0.92}

\lstdefinestyle{mystyle}{
    backgroundcolor=\color{backcolour},   
    commentstyle=\color{codegreen},
    keywordstyle=\color{magenta},
    numberstyle=\tiny\color{codegray},
    stringstyle=\color{codepurple},
    basicstyle=\ttfamily\footnotesize,
    breakatwhitespace=false,         
    breaklines=true,                 
    captionpos=b,                    
    keepspaces=true,                 
    numbers=left,                    
    numbersep=5pt,                  
    showspaces=false,                
    showstringspaces=false,
    showtabs=false,                  
    tabsize=2,
    language=[LaTeX]{TeX}
}
\lstset{style=mystyle}

\begin{document}
\title{\LaTeX-Für-Dullies}
\author{Max Schmidt}
\date{heute oder so}

\maketitle
\thispagestyle{empty}

\tableofcontents
\thispagestyle{empty}
\newpage

\section{Warum und wann \LaTeX?}
    \LaTeX ist super für längere Arbeiten bei denen es auf z.B. ein Inhaltsverzeichnis ankommt

\section{\LaTeX\ Editor Setup}

\section{Ein Dokument erstellen}
    Um ein Dokument zu erstellen einfach einen neuen Ordner erstellen und dann in VS Code
    eine neue Datei mit der Endung .tex erstellen.

asd

    \begin{lstlisting}
    \documentclass{article}
    \begin{document}
    Hier koennte Ihre Werbung stehen!
    \end{document}
    \end{lstlisting}
\section{Titelseite und Inhaltsverzeichnis}
\section{Sections und Co}
\section{Tabellen, Listen}
\section{Abblidungen}


\end{document}